\setchapterimage[6cm]{chapter/operating_system/logo.png}
\setchapterpreamble[u]{\margintoc}
\chapter[line 1 line 2]{What programming languages are operating systems written in}
\labch{operating-sysmets}

\section{Annotation}
The chapter explores the object of the <<operating system>> and its properties. The following problems were solved in the paper with the help of SPARQL queries: finding instances of the object <<operating system>>, building a list of operating systems (OS) by base, by creation time, by programming language, in which the OS was written. Also a histogram is constructed, it shows the number of programs written in some programming language, and the proportion of how many of them work for some OS. A lot of software does not specify the programming language on which it was developed. The property <<programming language>> was added to several objects to improve the results.

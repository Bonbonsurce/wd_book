\chapter*{Введение}
\label{ch:intro}

\newthought{Эта книга} предназначена для детей, родителей и учителей. 
Она познакомит вас с основами программирования, современными словами и понятиями в информатике\marginnote{%
todo about Computer science and Informatics
}. 
Прочитав книгу вы сможете извлекать из Викиданных информацию с~помощью SPARQL-скриптов, 
затем обрабатывать её и строить по ней таблицы, графики и карты.

Викиданные~--- это искусно сделанная база данных, которая как огромный кит лежит в основе громадной планеты Википедия\marginnote{%
todo об объёме Викиданных: ...
}. Впечатляет скорость роста этого кита, во многих главах мы будем обращать на это внимание.
Если изначально Викиданные создавались\marginnote{%
    todo about Дата создания Викиданных и организации, вложившиеся в создание проекта...
} %
для обслуживания нужд Википедии, 
то сейчас Викиданные используются крайне широко и в самых разных целях.

\newthought{Первая часть} книги включает в себя 10 уроков, описывающих язык программирования и протокол SPARQL.

\newthought{Вторая часть} содержит рецепты решения самых разных практических задач, 
возникающих при работе с объектами Викиданных.

\newthought{В третьей части} описано несколько исследований по Викиданным. 
Исследования выполнены студентами ПетрГУ, в том числе в рамках курса <<Программирование Викиданных>>, 
представленного на сайте Викиверситет\marginnote{%
todo add link
}. 
Этот курс растёт и пишется вместе с этой книгой. 

Потребовалось несколько лет работы со студентами в Википедии, прежде чем мы пришли с ними к таким проектам, 
как Викиверситет и Викиданные. Результатом работы в Википедии стало учебное пособие для тех, 
кто хочет научиться редактировать мировую энциклопедию\cite{Krizhanovsky2015}.


\newthought{Цель} этой книги быть учебником по Викиданным и языку SPARQL.

% \newthought{Книга научит} вас делать ... todo


Почему мы занялись и увлеклись Викиданными? 
Потому что это самая большая, сложная 
и быстрорастушая база данных на Земле. 
Потому что эта база лежит в основе Википедии, которую может редактировать каждый.
И Викиданные тоже может редактировать каждый. 

Работу серверов Википедии и Викиданных обеспечивают сотрудники организации Викимедиа. 
Не всё у разработчиков программ и сотрудников Викимедиа получается хорошо\marginnote{%
    %
    Сергей Рублёв: 
    <<Удивительно как что-то там работает вообще. На святом духе, как говорят русские.
    Когда Красоткин своих ботов разогнал, обнаружилось, как сказал Неолекс, что: 
    движок всей Викимедии это на сейчас $=\sim$ детский трёхколёсный велосипед с крыльями и реактивным двигателем. 
    На котором мы все (Википедию включая) отважно летим к звёздам. 
    В таком путешествии главное~--- ``не раскачивать лодку''.>> 
    \it{Сергей Рублёв, Александр Красоткин и Neolexx~--- 
    это вики-интеллегенция, то есть редакторы вики-проектов. 
    О ботах Викиновостей см. заметку} <<Фонд Викимедиа сломал Русские Викиновости>>, URL: \href{https://w.wiki/33d8}{w.wiki/33d8}.
    %
}, 
но другой Википедии у нас нет.

Будет ли в этой книге рассказано, как редактировать и пополнять Викиданные 
новой информацией? Нет, хотя это не сложнее, чем писать текст в SMS 
или статью в Википедии. В этом учебнике мы покажем вам, 
как задавать вопросы на языке этой базы. 
Чтобы вы ориентировались в ней лучше, чем у себя на кухне. 

Почему школьники? 
Потому что вообще программирование и программирование Викиданных в частности~--- это 
увлекательная и, если увлечься, простая вещь. 
Если с помощью учителя или самостоятельно школьник увлечётся, 
то сможет программировать Викиданные, писать статьи Википедии. 
Есть десятки школьников, плодотворно работающих в Википедии. 
Надеемся, что благодаря этой книги одним из первых языков, 
изучаемых по информатике в~школе, станет язык Викиданных.

Покажем прямо сейчас, что Викиданные находятся от нас на расстоянии одного клика.
Откройте на компьютере или на телефоне такую ссылку: 
\href{https://w.wiki/jcE}{https://w.wiki/jcE}. 
Вы увидите главное окно, в котором мы с вами будем писать наши небольшие програмы. 
Если вы нажмёте большую синюю кнопку с белым треугольником, 
    \marginnote[0cm]{
    TODO Добавить картинку с кнопкой на поля
    }
то запустите 
эту программу из шести строк. Программа обратится к базе данных 
и спросит, какие города Земли есть в Викиданных?
Результатом будет список городов. 
Подробный рассказ о том, что можно узнать и подсчитать о городах в Викиданных,  
читайте в отдельной главе на с.~\pageref{ch:city}. 

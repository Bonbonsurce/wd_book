\chapter{Населённые пункты}
\label{ch:human-settlement}

	Викиданные представляют собой базу структурированных данных. В статье исследуется объект Викиданных <<населённый пункт>> (human settlement) и его свойства. 
В каждом из разделов представлены задачи, решённые с помощью SPARQL-запросов. В их числе: нахождение экземпляров объекта <<населённый пункт>>, построение
упорядоченного списка стран по суммарному количеству населения, проживающего в <<населённых пунктах>>, и списка объектов, сопутствующих <<населённым 
пунктам>> в свойстве <<экземпляр>> (instance of). Также построена диаграмма, показывающая долю населения страны, проживающего в <<населённых пунктах>>. 
Диаграмма показывает, что высокий процент населения, проживающего в <<населённых пунктах>>, приходится на менее промышленные страны, в то время как в более 
индустриальных странах меньшая доля населения проживает в <<населённых пунктах>>. На 2017 год Википедия описывает примерно половину населённых пунктов 
(75 тыс.), Викиданные содержат менее 3\% таких поселений (4 тыс.) относительно данных переписи за 2010 год (155,5 тыс.). Для улучшения результатов решения 
вышеописанных задач находили более общие объекты и указывали их в исследуемом объекте с помощью свойства "экземляр" (instance of). Трудность исследования 
вызвана отсутствием чёткой типологии населённых пунктов (например, от численности населения) в законодательстве России и в Викиданных.

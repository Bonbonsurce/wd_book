\chapter{Музыкальные композиции}
\label{ch:musical-composition}
Статья посвящена исследованию музыкальных композиций на основе базы знаний международного проекта Викиданные. С помощью SPARQL-запросов, вычисляемых на объектах типа ”музыкальная композиция” в Викиданных, получен список всех музыкальных композиций, список музыкальных композиций, имеющих композиторов, а также построена пузырьковая диаграмма, показывающая композиторов с наибольшим количеством композиций. Кроме того, решена задача поиска музыкальных лакун в общественном достоянии и выполнена оценка полноты Викиданных.

\section{Экземпляры объекта «Музыкальные композиции»}

\marginnote {Используемые в запросах объекты: \item\wdqNameqwsa {<<музыкальная композиция>>} {207628};
Используемое свойство: \wdProperty{31}{<<экземпляр>>}
}
Построим список всех музыкальных композиций.

\begin{lstlisting}[ language=SPARQL, numbers=none,
                    caption={\href{https://w.wiki/52VP}{Список всех  музыкальных композиций}\protect\footnotemark},
                    label=lst:musical-composition,
                    texcl 
                    ]
#List of all musical compositions
SELECT ?composition ?compositionLabel 
WHERE {
  ?composition wdt:P31 wd:Q207628.
  SERVICE wikibase:label { bd:serviceParam wikibase:language "ru". }
}
\end{lstlisting}%

\footnotetext{Ссылка на SPARQL-запрос: \href{https://w.wiki/52VP}{https://w.wiki/52VP}. 5494 записи в 2017 году.}
Наиболее полными и проработанными музыкальными композициями на Викиданных являются: \wdqName{Волшебная флейта}{5064}, \wdqName{К Элизе}{11980}, \wdqName{Реквием}{207875}, \wdqName{Маленькая ночная серенада}{12025}.

Почти пустыми и малоинформативными музыкальными композициями были: \wdqName{Полёт шмеля}{1342275}, \wdqName{Ромео и Джульетта}{763716}, \wdqName{Симфонический эпизод «Завод»}{1845909}, \wdqName{Binks’ Waltz}{28807544}, \wdqName{The Rose-bud March}{28803595}, \wdqName{Leola}{28804177}.

В 2022 году тот же скрипт нашёл 453 музыкальные композиции вместо 5,5 тысяч в 2017 году. Уменьшение числа композиций связано с тем, что эти объекты Викиданных являются теперь не экземплярами объекта «музыкальные произведения», а его экземплярами различных подклассов «музыкального произведения». При поиске подклассов объекта «музыкальные произведения» можно найти такие жанры: \wdqName{драматико-музыкальное произведение}{58483083}, \wdqName{гимн}{484692}, \wdqName{баллада}{182659}.

Найдём количество музыкальных композиций в каждом жанре с помощью следующего запроса.

\begin{lstlisting}[ language=SPARQL, numbers=none,
                    caption={\href{https://w.wiki/4$PC}{ Количество музыкальных композиций в каждом жанре}\protect\footnotemark},
                    label=lst:subjects-of-Russia,
                    texcl 
                    ]
# Count of pieces of music in each subclass
SELECT ?type (COUNT(?subMusicInstance) AS ?count) ?subMusicLabel WHERE {
  ?type wdt:P279* wd:Q207628.      # subclass of musical composition
  ?subMusicInstance wdt:P31 ?type  # instance  of that class of which this subject is a particular example and member
  SERVICE wikibase:label { bd:serviceParam wikibase:language "ru, en". }
}
GROUP BY ?type ?subMusicLabel
ORDER BY DESC (?count)
\end{lstlisting}%

\footnotetext{Ссылка на \href{https://w.wiki/4$PC}{SPARQL-запрос}, 232 подкласса музыкальных композиций на 2022 год.}

Теперь подсчитаем общее суммарное число музыкальных произведений с учётом музыкальных композиций в подклассах. Для этого добавим в наш скрипт команду SUM() и удалим лишние строки. Получим такой код:

\begin{lstlisting}[ language=SPARQL, numbers=none,
                    caption={\href{https://w.wiki/4$Q$}{ Суммарное число музыкальных произведений с учётом музыкальных композиций в подклассах}\protect\footnotemark},
                    label=lst:oblast-of-Russia,
                    texcl 
                    ]
# The total number of musical works for all subclasses 
SELECT (SUM(?count) AS ?sum) WHERE{
  SELECT (COUNT(?music) AS ?count) WHERE {
    ?type wdt:P279* wd:Q207628.  # subclass of musical composition
    ?music wdt:P31 ?type  # instance  of that class of which this subject is a particular example and member
  }
}
\end{lstlisting}%

\footnotetext{Ссылка на \href{https://w.wiki/4$Q$}{SPARQL-запрос}.

Можно записать этот код еще короче. Переменная ?type нам не нужна, поэтому можно обойтись без неё, а 4 и 5 строки поменяем местами.

\begin{lstlisting}[ language=SPARQL, numbers=none,
                    caption={\href{https://w.wiki/4$R3}{ Суммарное число музыкальных произведений с учётом музыкальных композиций в подклассах}\protect\footnotemark},
                    label=lst:oblast-of-Russia,
                    texcl 
                    ]
# The total number of musical works for all subclasses 
SELECT (SUM(?count) AS ?sum) WHERE{
  SELECT (COUNT(?music) AS ?count) WHERE {
    ?music wdt:P31   # instance of
          [ wdt:P279* wd:Q207628 ]. # subclass of musical composition
  }
}
\end{lstlisting}%

\footnotetext{Ссылка на \href{https://w.wiki/4$R3}{SPARQL-запрос}, на 1 апреля 2022 года запрос выдает 135466 музыкальных произведений. По сравнению с 2017 годом "5494 записи", число записей увеличилось в несколько раз. Это связано с тем, что за 5 лет было добавлено множество новых музыкальных произведений, а также старых, которые не учли ранее.}

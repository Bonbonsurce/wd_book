\chapter{Ответы}
\label{ch:answers}

\openepigraph{%
Ставлю три звездочки. 
Я видал в детских книжках: 
когда человек делает прыжок к новой мысли, он ставит три звездочки...
}{Саша Чёрный, {Дневник Фокса Микки}
}

\newthought{Задачи рассеяны} по всей книге, а ответы на них собраны в этом разделе.

\begin{task}
    \label{answer:instance-in-OOP-vs-Wikidata}
    \newthought{Экземпляр объекта}\footnote[][0cm]{См. статью 
            \href{https://en.wikipedia.org/wiki/Instance\_(computer\_science)}{Instance (computer science)} в Английской Википедии.
            }
    в Викиданных 
    и в объектно-ориентированном программировании (ООП) сходны в сути, а именно:
    есть модель базового объекта $D$, создаётся новая единица $I$, обладающая 
    свойствами той же модели $D$. 
    В программировании говорят, что класс $D$ проинициализирован 
    и получен объект $I$\footnote[][0cm]{$I$ is instance of $D$}.

    В чём разница? 
    В ООП в исходном коде программы мы видим как во времени последовательно 
    (в разных строках программы) происходит 
    объявление переменной, инициализация класса, 
    присвоение значений экземплярам класса.
    В Викиданных в тот момент, когда выполняется скрипт и происходит обращение 
    к данным, объекты, являющиеся экземплярами других объектов, 
    уже представлены и обычно не происходит их изменение, 
    связанное с работой SPARQL-скриптов.

    \small{Вопрос на с.~\pageref{question:instance-in-OOP-vs-Wikidata}.}
\end{task}


\begin{task}
    \label{answer:guess_numbers_task}
    \newthought{В алгоритме угадывания чисел} число $ a $ может быть натуральным, целым, вещественным и рациональным числом, то есть $ a \in \mathbb{N},\mathbb{Z}, \mathbb{Q}, \mathbb{R} $, но кроме иррациональных чисел из-за конечности разрядной сетки компьютера. 
    \small{Вопрос на с.~\pageref{question:text}}
\end{task}

\begin{task}
    \label{answer:global-vars-pros-cons}
    \newthought{Ответ такой} ... todo. 

    \small{Вопрос на с.~\pageref{fig:block:proc:swap:colors}.}
\end{task}

%%%%%%%%%%%%%% City chapter %%%%%%%%%%%%%%
\begin{task}
    \label{answer:cities_geographic_objects}
    \newthought{В честь географических объектов} были названы Тула (\href{https://ru.wikipedia.org/wiki/Тулица}{река Тулица}), Курильск (\href{https://ru.wikipedia.org/wiki/Курильские_острова}{Курильские острова}) и Вологда (\href{https://ru.wikipedia.org/wiki/Вологда_(река)}{река Вологда}). Ответ на вопрос также можно получить, выполнив следующий SPARQL-запрос (листинг \ref{lst:cities_geographic_objects}). Значение свойства \href{https://www.wikidata.org/wiki/Property:P138}{named after} показывает, в честь какого объекта Викиданных был назван город.
    \begin{lstlisting}[ language=SPARQL, 
                    caption={\href{https://w.wiki/jiu}{Города, названные в честь географических объектов}},
                    label=lst:cities_geographic_objects, 
                    escapebegin=ку,escapeend=ку-ку>
                    ]
SELECT ?city ?cityLabel ?namedAfterLabel ?whatIsItLabel WHERE {
	?city wdt:P31/wdt:P279* wd:Q7930989.
	?city wdt:P138 ?namedAfter.
	?namedAfter wdt:P31 ?whatIsIt.
	FILTER(?city = wd:Q1341 || ?city = wd:Q2770 || ?city = wd:Q5655 ||
		?city = wd:Q156046 || ?city = wd:Q1957 || ?city = wd:Q175651)
	SERVICE wikibase:label { bd:serviceParam wikibase:language "ru" }
}
    \end{lstlisting}
    \small{Вопрос на с.~\pageref{lst:population_town}.}
\end{task}

\begin{task}
    \label{answer:cities_over_400_age}
    \newthought{Более 400 лет назад} были основаны Москва (1147 г.), Воронеж (1586 г.), Самара (1586 г.), Казань (1005 г.) и Астрахань (1558 г.). Ответ на вопрос также можно получить, выполнив следующий SPARQL-запрос (листинг \ref{lst:cities_over_400_age}). Значение свойства \href{https://www.wikidata.org/wiki/Property:P571}{inception} содержит дату основания города.
    \begin{lstlisting}[ language=SPARQL, 
                    caption={\href{https://w.wiki/jnv}{Города, основанные более 400 лет назад}},
                    label=lst:cities_over_400_age, 
                    escapebegin=ку,escapeend=ку-ку>
                    ]
SELECT ?city ?cityLabel ?inceptionDate WHERE {
	?city wdt:P31/wdt:P279* wd:Q7930989.
	?city wdt:P17 wd:Q159.
	?city wdt:P571 ?inceptionDate.
	FILTER(BOUND(?inceptionDate) && 
					DATATYPE(?inceptionDate) = xsd:dateTime).
	BIND(NOW() - ?inceptionDate AS ?distance).
	FILTER(0 <= ?distance && ?distance > 146000). # = 400 * 365
	FILTER(?city = wd:Q649 || ?city = wd:Q193522 || ?city = wd:Q900 ||
		?city = wd:Q3927 || ?city = wd:Q894 || ?city = wd:Q3426)
	SERVICE wikibase:label { bd:serviceParam wikibase:language "ru" }
}
GROUP BY ?city ?cityLabel ?inceptionDate    
\end{lstlisting}

    \small{Вопрос на с.~\pageref{lst:population_city_millions}.}
\end{task}

%\begin{task}
%    \label{answer:cities_flags}
%    \newthought{Флаг, изображенный на рис. } принадлежит .  
%    \small{Вопрос на с.~\pageref{fig:Pobeda-beda}.}
%\end{task}
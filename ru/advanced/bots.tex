\chapter{Боты в Викиданных}
\label{ch:bots}
В этой главе рассматривается автоматизация процессов в Викиданных. Во многих случаях мы хотим исправить повторяющиеся ошибки или ввести большие объемы данных в Викиданные вместо того, чтобы изменять свойства по одному. Для ввода данных в нашем распоряжении имеется несколько инструментов, облегчающих нашу работу, таких как OpenRefine или QuickStatements, но повторяющиеся изменения и вставки со временем должны выполняться с помощью бота.


\section{Требования}
\label{sec:requirements}
Мы можем запрограммировать бота на нескольких языках программирвания, но, чтобы облегчить нашу задачу, мы можем использовать Pywikibot, набор инструментов, запрограммированных на Python для облегчения доступа к информации в проектах Фонда Викимедиа. У нас есть три варианта запуска наших программ:
\begin{enumerate}
  \setlength{\itemindent}{2em}
  \item использовать веб-оболочку, такую как PAWS,
  \item создать облаччную инфраструктуру, такую как Toolforge, или 
  \item запустить на нашем собственном компьютере.
\end{enumerate}

Любой из этих трех вариантов может быть использован благодаря объяснениям, которые есть в \href{https://www.mediawiki.org/wiki/Manual:Pywikibot/Installation}{\color{blue}официальной документации}. Если мы используем собственный компьютер, мы должны установить Python, следуя инструкциям в документации. После установки мы можем проверить, корректно ли он установлен, нажав <<py -V>> в командной строке Windows или в консоли Linux (в дальнейшем все должно быть сделано отсюда), и он покажет нам версию, с которой мы будем работать. Затем мы должны установить Pywikibot и настроить его. Эта конфигурация состоит из нескольких этапов:
\begin{itemize}
  \setlength{\itemindent}{2em}
  \item Выполните следующую команду, чтобы создать файл конфигурации: <<py pwb.py generate\_user\_files>>.
  \item Выполните следующую команду для входа в систему: <<py pwb.py login>>.
  \item Просмотрите файлы user-config.py и user-password.py, где мы можем указать имена пользователей, которые будут вносить изменения.
\end{itemize}

Для просмотра информации или выполнения небольших тестов мы можем использовать наше имя пользователя, но в случае внесения большого количества изменений в Викиданные рекомендуется создать учетную запись бота и запросить разрешение на запуск. Чтобы иметь этот флаг, обычно требуются некоторые знания о редактировании в проектах Викимедиа, которые подтверждают, что мы не сделаем серьезных ошибок и у нас есть поддержка других пользователей.

Обычно для получения этого флага нас просят выполнить определенное количество тестовых выпусков, чтобы удостовериться, что такие изменения имеют смысл, поскольку у нас должен быть низкий или нулевой коэффициент ошибок. Мы также можем выполнять задачи через нашу учетную запись, но с небольшим количеством правок.

\section{Наш первый скрипт}
\label{sec:firstScript}
После того, как мы правильно настроили Pywikibot, мы можем запустить наш первый скрипт. Перейдем в папку, в которой находится Pywikibot, и создадим новый файл с именем test.py.

Мы можем вносить изменения в файл при помощи любых текстовых редакторов или редакторов кода, таких как Visual Studio Code, Notepad++ или обычного Блокнота. Оказавшись внутри файла, поместим следующие строки кода:
\definecolor{codegray}{rgb}{0.5,0.5,0.5}
\definecolor{codepurple}{rgb}{0.58,0,0.82}
\definecolor{backcolour}{rgb}{0.95,0.95,0.92}

\lstdefinestyle{mystyle}{
  backgroundcolor=\color{backcolour},
  keywordstyle=\color{magenta},
  numberstyle=\tiny\color{codegray},
  stringstyle=\color{codepurple},
  basicstyle=\ttfamily\footnotesize,
  numbers=left,                    
  numbersep=4pt,                  
}
\lstset{style=mystyle}
\begin{lstlisting}[language=Python]
import pywikibot

site=pywikibot.Site('wikidata', "wikidata")

page=pywikibot.Page(site, u"Q16583338")

print(page.get())
\end{lstlisting}
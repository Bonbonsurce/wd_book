\chapter{Ответы}
\label{ch:answers}

\openepigraph{%
Ставлю три звездочки. 
Я видал в детских книжках: 
когда человек делает прыжок к новой мысли, он ставит три звездочки...
}{Саша Чёрный, {Дневник Фокса Микки}
}

\newthought{Задачи рассеяны} по всей книге, а ответы на них собраны в этом разделе.

\begin{task}
    \label{answer:instance-in-OOP-vs-Wikidata}
    \newthought{Экземпляр объекта}\footnote[][0cm]{См. статью 
            \href{https://en.wikipedia.org/wiki/Instance\_(computer\_science)}{Instance (computer science)} в Английской Википедии.
            }
    в Викиданных 
    и в объектно-ориентированном программировании (ООП) сходны в сути, а именно:
    есть модель базового объекта $D$, создаётся новая единица $I$, обладающая 
    свойствами той же модели $D$. Говорят, что модель (класс) $D$ проинициализирована 
    и получен объект $I$\footnote[][0cm]{$I$ is instance of $D$}.

    В чём разница? 
    В ООП в исходном коде программы мы видим как во времени последовательно 
    (в разных строках программы) происходит 
    объявление переменной, инициализация класса, 
    присвоение значений экземплярам класса.
    В Викиданных в тот момент, когда выполняется скрипт и происходит обращение 
    к данным, объекты, являющиеся экземплярами других объектов уже представлены 
    и обычно не происходит их изменение, связанное с работой SPARQL-скриптов.

    \small{Вопрос на с.~\pageref{question:instance-in-OOP-vs-Wikidata}.}
\end{task}

\begin{task}
    \label{answer:Pobeda-beda}
    \newthought{Название корабля укоротилось} в мультфильме <<Приключения капитана 
    Врунгеля>>, поскольку юнга Лом срубил свежий лес для постройки яхты, поэтому корабль 
    и прирос к берегу. Пришлось рубить корни, первые две буквы не удержались и отвалились, 
    яхта <<Победа>> превратилась в <<Беду>>. 
    \small{Вопрос на с.~\pageref{fig:Pobeda-beda}.}
\end{task}

Задача: сколько ссылок можно закодировать, если ссылки содержит ровно 7 буквенно-цифровых символов,
     каждый символ может быть буквой английского алфавита 
     или цифрой от 0 до 9? Ответ на странице ... todo 

temp: Чем хороши глобальные переменные и какие у них недостатки
по сравнению с локальными переменными?

\begin{task}
    \label{answer:guess_numbers_task}
    \newthought{В алгоритме угадывания чисел} число $ a $ может быть натуральным, целым, вещественным и рациональным числом, то есть $ a \in \mathbb{N},\mathbb{Z}, \mathbb{Q}, \mathbb{R} $, но кроме иррациональных чисел из-за конечности разрядной сетки компьютера. 
    \small{Вопрос на с.~\pageref{question:text}}
\end{task}

\begin{task}
    \label{answer:global-vars-pros-cons}
    \newthought{Ответ такой} ... todo. 

    \small{Вопрос на с.~\pageref{fig:block:proc:swap:colors}.}
\end{task}

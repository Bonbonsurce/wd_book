\setchapterpreamble[u]{\margintoc}
\chapter{Where do inventors of programming languages study and what are their profession}
\labch{programming_language}

\section{Abstract}
The article examines the properties of programming languages based on the knowledge base of the international project Wikidata. A number of problems have been solved with the help of SPARQL queries calculated on objects of the "programming language"\  type in Wikidata. Lists of all programming languages under permissive licenses and languages with closed licenses were obtained and their percentage was calculated. A bubble chart was built by the number of source code file formats. Maps have been received showing the location of educational institutions and companies in which people associated with the creation of programming languages studied or worked. A bubble diagram has been built showing the professions of people involved in the creation and development of programming languages. A list of all object-oriented programming languages was obtained and a conclusion was drawn about the exhaustive completeness of Wikidata regarding them. Comparison and analysis of the results of SPARQL queries of 2017 and 2020 are carried out, the main changes are noted.


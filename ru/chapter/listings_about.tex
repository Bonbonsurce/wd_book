\chapter{Программы и программки в книге}
\label{ch:listing_about}

В книге будет приводиться \emph{программный код}\footnote[][0cm]{
    Программный код также называют  \emph{исходным кодом} или 
    \emph{листингом}.%\url{http://bit.ly/2OgIdWo}
%   
} на языке SPARQL. 
Именно на этом языке пишут запросы к Викиданным.


Вот пример SPARQL-скрипта (листинг~\ref{lst:cities}), 
с помощью которого можно получить из Викиданных список городов, 
точнее экземляров объекта  
\wdqName{city}{515}.

\begin{lstlisting}[ language=SPARQL, 
                    caption={\href{https://w.wiki/jcE}{Список городов}},
                    label=lst:cities, 
                    escapebegin=ку,escapeend=ку-ку>
                    ]
SELECT ?city ?cityLabel WHERE {
  ?city wdt:P31 wd:Q515.       # instance of city
  SERVICE wikibase:label { bd:serviceParam wikibase:language "ru" }
}
\end{lstlisting}


Мы будем регулярно ссылаться на объекты Викиданных. 
Например, \wdqName{city}{515}\footnote[][0cm]{%
%    
    В электронной версии книги имена объектов Викиданных --- 
    это гиперссылки на соответствующие страницы Викиданных.
} 
--- это имя объекта Викиданных. 
Здесь ``city'' имя метки объекта (Label), 
a ``Q515`` -- это уникальный идентификатор объекта 
и название страницы Викиданных с описанием этого объекта.
\href{https://en.wikipedia.org/wiki/Rule_of_thumb}{Rule of thumb}\footnote[][0cm]{%
%
        \TODO{ О значении и про этимологию из Викисловаря...}
} 
для номеров объектов Викиданных таков, что более значимые объекты 
(например, Солнце --- идентификатор \wdq{525}) 
имеют меньший номер, чем менее известные
(мушка дрозофилы --- \wdq{312154}).

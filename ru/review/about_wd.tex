\chapter{Обзор Викиданных}
\label{ch:ReviewAboutWD}

\section{Викиданные}
Викиданные — это структурированная и совместно редактируемая база данных, созданная Фондом Викимедиа. Проект был официально запущен 30 октября 2012 года, его разработка ведется под руководством Wikimedia Deutschland\cite{Wikidata_review}. Проект создавался за счёт пожертвований Allen Institute for Artificial Intelligence, Gordon and Betty Moore Foundation и Google. В данный момент Викиданные — это бесплатная и свободная база знаний, которая может использоваться и редактироваться людьми и машинами\cite{Vrandecic}.

Любой объект Викиданных имеет свой уникальный идентификатор и свойства. Эта информация может быть обработана с помощью компьютера, и при этом она понятна пользователям. Сайт Викиданных содержит сервис «Wikidata Query», включающий набор инструментов для построения SPARQL-запросов и их визуализации в виде таблиц, диаграмм, графов или географических карт.

Содержимое Викиданных распространяется по лицензии Creative Commons CC0, которая позволяет повторно использовать информацию самыми разными способами: пользователи могут копировать, изменять, распространять и обрабатывать эти данные в любых целях. Еще одна особенность Викиданных  это многоязычность. Любой человек может редактировать Викиданные более чем на 350 языках.

Викиданные постоянно обновляются, добавляются новые объекты. Сейчас насчитывается более 63 миллионов страниц и более 883 миллионов правок. 15 марта 2019 года в Викиданных была совершена 883 173 631 правка, что превзошло количество правок в английской Википедии и сделало Викиданные наиболее редактируемым сайтом Викимедиа\cite{Wikipedia_review}.
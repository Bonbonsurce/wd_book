\setchapterimage[6cm]{chapter/anime/Wikipe-tan-box.png}

\chapter{Anime: a mysterious and stunning world of Japanese animation\protect\footnotemark}

\labch{anime}

\footnotetext{\href{https://en.wikipedia.org/wiki/Wikipedia:Wikipe-tan}{Wikipe-tan}, unofficial anime mascot of Wikipedia. 
Author: \href{https://commons.wikimedia.org/wiki/File:Wikipe-tan_cropped_itembox.png}{Waihorace, WikiCommons / 2011 / Creative Commons Attribution-Share Alike 3.0 Unported License}.}

This article is dedicated to anime Wikidata object analysis. Using SPARQL queries executed on Wikidata objects of anime type, several tasks were accomplished. The list of seiyu ordered by number of anime voiced by them is shown, the histogram of seiyu that acted in one or more anime is presented, the graph that connects seiyu and anime voiced by them is constructed.